\vspace{6pt}

\subsection{Other Writing}

\vspace{5pt}

\begin{itemize}

    \setlength\itemsep{1em}
    
    \item Engineering Software as a Service. 2nd. Edition Fox, Armando. Contributed Book Chapter on Accessibility.
    \newline
    \href{https://saasbook.info}{https://saasbook.info}

    \item{\textit{AUTOGRADING FOR SNAP!} Hello World Magazine, Issue 3, Autumn 2017}
    \newline
    \small{\href{https://magazines-static.raspberrypi.org/issues/full\_pdfs/000/000/004/orignal/HelloWorld03.pdf}{https://magazines-static.raspberrypi.org/issues/full\_pdfs/000/000/004/orignal/HelloWorld03.pdf}}
    
    \item{"Where are the Practical Computing Classes?" The Daily Califonian, Op-Ed Jan. 31, 2014
    \newline
    \href{https://www.dailycal.org/2014/01/31/practical-computing-classes/}{https://www.dailycal.org/2014/01/31/practical-computing-classes/}}

\end{itemize}


\subsection{Theses}

\vspace{5pt}

\begin{itemize}

  \setlength\itemsep{1em}
  
  \item \textbf{Masters Thesis, 2016:} \textit{Lambda: Autograding For Snap!}

    % TODO: Shrink and move up.
    \small{Lambda is an autograding platform for Snap!, a blocks-based programming language. As an undergraduate, I contributed to designing the system, architecting code, and creating autograder tests for student projects. During my Master's Project, I created a server-side component for the autograder to support using LTI (Learning Tools Interoperability), an open educational standard, which allowed embedding the autograder into UC Berkeley's LMS. The autograder was used to trial a series of in-lab exercises, in comparison to the oral lab check-off questions students were answering.}
    \linebreak
    \small{\href{http://www2.eecs.berkeley.edu/Pubs/TechRpts/2018/EECS-2018-2.pdf}{http://www2.eecs.berkeley.edu/Pubs/TechRpts/2018/EECS-2018-2.pdf}}

\end{itemize}

\vspace{2pt}