\section{Teaching — UC Berkeley}

\vspace{6pt}

\subsection{Course Descriptions}

\vspace{5pt}
\begin{description}
    \setlength\itemsep{0.7em}
    \item[CS 10] \textit{The Beauty and Joy of Computing} is a non-majors introduction to computer science. This is a hands-on project-based course taught using \snap and Python.  (4 units) % I contribute to the infrastructure and externally facing curriculum even not teaching.
    
    \item[CS 88 / DATA C88C] \textit{Computational Structures for Data Science} is a CS1 or "1.5" course for data science majors. Based on Sussman's venerable SICP book, taught in Python. (3 units)

    \item[CS 169] \textit{Engineering Software as a Service} The original software engineering course, later split into CS169A and CS169L. (4 units)
    
    \item[CS169A] \textit{Engineering Software as a Service} is an upper division course which focuses on the dynamics of modern software engineering through RESTful design, HTTP, and team projects. (4 units)
    
    \item[CS169L] \textit{Software Engineering Team Project} is a follow up CS169L, where teams of students get hands-on experience building a semester-long project for a campus organization or local nonprofit.  (4 units)
    
    \item[CS W186] \textit{Introduction to Databases} is an upper division databases course, offered in an online format.  (4 units)
    
    \item[CS 194-23] \textit{Art \& Science of Digital Photography} is a highly technical photography course, bridging both art and engineering. (4 units)
    
    \item[CS 195 / CS H195] \textit{The Social Implications of Computing Technology} is a course for students to explore the unintended consequences of computing, including the culture of how we build programs. H195 is an honors course where students complete a small research project. (1 unit / 3 units)
    
    \item[CS 294-188] \textit{Design and Evaluation of CS at Scale} is a pedagogy course for teaching assistants designed to engage in course-course collaborations and learn about current computing education pedagogy.  (1 unit)
    
    \item[CS 302] \textit{Designing Computer Science Education} is a course which students take to prepare them to be the instructor of record for a summer CS course. (3 units)
    
    \item[CS 375] \textit{Teaching Practicum} is the course all first time teaching assistants in computer science take. (2 units)

    \item[DATA 101 / CS C187] \textit{Data Engineering} an upper division course putting databases into practice. Topics include SQL, window functions, indexing, design, ETL, NoSQL, graph databases, etc. (4 units)
\end{description}

\vspace{5pt}

\subsection{Teaching Assignments}

\textbf{Notes}:
\begin{itemize}
    \item \texttt{*} denotes courses which were co-taught.
    \item Spring 2020 included an abrupt switch to remote instruction.
    \item Fall 2020, Spring 2021 were entirely remote semesters.
    \item Fall 2021, Spring 2022, and Fall 2022 included significant hybrid teaching components.
\newline

\end{itemize}

\newcommand{\semester}[2]{\item{\textbf{#1}\newline}#2}

\begin{itemize}
    \setlength\itemsep{0.7em}

    \semester{Expected Fall 2024}{DATA 101* (450 students), CS 169A (280 students)}

    \semester{Summer 2024}{Co-Organizer EECS Summer Program (11 courses)}
    
    \semester{Spring 2024} {DATA C88C (500 students, 14 Staff), CS 169L* (16 students, 2 staff), CS 302* (20 students, 0 staff)}
    
    \semester{Fall 2023} {DATA C88C (460 students, 10 staff), CS 169A* (270 students, 6 staff)}

    \semester{Summer 2023}{Organizer EECS Summer Program (10 courses)}

    \semester{Spring 2023} {DATA C88C (500 students, 15 staff), CS 169L* (20 students, 1 TA), CS 302 (24 students, 0 staff)}
    
    \semester{Fall 2022}{CS 88 (450 students, 14 staff), CS 169A* (240 students, 7 staff), CS 195/H195* (300 students, 1 TA)} 

    \semester{Summer 2022}{Co-Organizer EECS Summer Program (10 courses)}

    \semester{Spring 2022} {CS 88 (420 students, 12 staff), CS 10 (150 students, 6 staff), CS 302* (25 students, 1 reader)}

    \semester{Fall 2021} {CS 88 (400 students), CS 169A* (300 students), CS294-188 (10 students)}

    \semester{Spring 2021} {CS 88 (400 students), CS 195 (400 students), CS169L* (60 students)}

    \semester{Fall 2020} {CS 88 (360 students), CS 169A* (300 students)}

    \semester{Spring 2020} {CS 88* (300 students), CS W186* (700 students), CS 195* (400 students)}
      
    \semester {\textbf Fall 2019} {CS 88 (240 students, 8 staff), CS 169 (120 students, 4 staff), CS 375 (80 students)}

    \semester {Summer 2015} CS 10* (60 students, 5 staff)

    \semester {Spring 2015} {Head-TA, CS 10 (330 students)}
    
    \semester {Fall 2014} {Head-TA, CS 10 (330 students)}
    
    \semester {Spring 2014} {Head-TA, CS 10 (300 students)}
    
    \semester {Fall 2013} {TA, CS 10 (300 students)}
    
    \semester {Spring 2013} {TA, CS 194-23: The Art \& Science of Digital Photography (30 students); TA, CS 10 (300 students)}
    
    \semester {Fall 2012} {TA, CS 10 (240 students)}
\end{itemize}
