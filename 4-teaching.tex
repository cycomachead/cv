\section{Teaching — UC Berkeley}

\vspace{6pt}

\subsection{Course Descriptions}

\vspace{5pt}
\begin{description}
    \setlength\itemsep{0.7em}
    \item[CS10] \textit{The Beauty and Joy of Computing} is a non-majors introduction to computer science. This is a hands-on project-based course taught using \snap and Python.  (4 units) % I contribute to the infrastructure and externally facing curriculum even not teaching.
    \item[CS88 / DATA C88C] \textit{Computational Structures for Data Science} is a CS1 or "1.5" course for data science majors. It is based on the venerable SICP book, taught in Python. (3 units)

    \item[CS169] \textit{Engineering Software as a Service} The original software engineering course, later split into CS169A and CS169L. (4 units)
    
    \item[CS169A] \textit{Engineering Software as a Service} is an upper division course which focuses on the dynamics of modern software engineering through RESTful design, HTTP, and team projects. (4 units)
    
    \item[CS169L] \textit{Software Engineering Team Project} is a follow up CS169L, where teams of students get hands-on experience building a semester-long project for a campus organization or local nonprofit.  (4 units)
    
    \item[CSW186] \textit{Introduction to Databases} is an upper division databases course, offered in an online format.  (4 units)
    % \item[CS194-26] 
    \item[CS195 / CSH195] \textit{The Social Implications of Computing Technology} is a course for students to explore the unintended consequences of computing, including the culture of how we build programs. H195 is an honors course where students complete a small research project. (1 unit / 3 units)
    
    \item[CS294-188] \textit{Design and Evaluation of CS at Scale} is a pedagogy course for TAs designed to engage in course-course collaborations and learn about current computing education pedagogy.  (1 unit)
    \item[CS302] \textit{Designing Computer Science Education} is a course which students take to prepare them to be the instructor of record for a summer CS course. (3 units)
    
    \item[CS375] \textit{Teaching Practicum} is the course all first time teaching assistants in computer science take. (2 units)
\end{description}

\vspace{5pt}

\subsection{Teaching Assignments}

\textbf{Notes}:
\newline
\texttt{*} denotes courses which were co-taught.
\newline
Spring 2020 included an abrupt switch to remote instruction.
\newline
{Fall 2020, Spring 2021 were entirely remote semesters.}
\newline
{Fall 2021, Spring 2022, and Fall 2022 included significant hybrid teaching components.}
\newline

% \begin{itemize}
    % \setlength\itemsep{0.7em}

    \cvitem{\textbf Spring 2024}{DATA C88C (500 Students, 14 Staff), CS 169L* (16 students, 2 staff), CS 302* (20 students, 0 staff)}
    
    \cvitem{\textbf Fall 2023}{DATA C88C (460 Students, 10 Staff), CS 169A* (270 Students, 6 Staff)}
    
    \cvitem {\textbf Spring 2023} {DATA C88C (500 Students, 15 Staff), CS 169L* (20 Students, 1 TA), CS 302 (24 Students, 0 TAs)}
    
    \cvitem {\textbf Fall 2022} CS 88 (450 students, 14 staff), CS 169A* (240 students, 6 TAs), CS 195/H195* (300 students, 1 TA) 
    
    \cvitem {\textbf Spring 2022} CS 88 (420 students, 12 staff), CS 10 (150 students, 6 staff), CS 302* (25 students, 1 reader)

    \cvitem {\textbf Fall 2021} CS 88 (400 students), CS 169A* (300 students), CS294-188 (10 students)

    \cvitem {\textbf Spring 2021} CS 88 (400 students), CS 195 (400 students), CS169L* (60 students)

    \cvitem {\textbf Fall 2020} CS 88 (360 students), CS 169A* (300 students)

    \cvitem {\textbf Spring 2020} CS 88* (300 students), CS W186* (700 students), CS195* (400 students)
      
    \cvitem {\textbf Fall 2019} CS88 (240 students), CS169 (120 students), CS375 (80 students)
    % 240 Students, 8 TAs
    % 120 Students, 3 TAs
    % 80 Students, 1 TA
    \cvitem {\textbf Summer 2015} CS10* (60 students)
    % (non-majors CS course)
    % \small Instructor of Record for 60 students and 6 staff members. I oversaw all aspects of the course and gave lectures.
    \cvitem {\textbf Spring 2015} Head-TA, CS10 (330 students)
    \cvitem {\textbf Fall 2014} Head-TA, CS10 (330 students)
    
    \cvitem {\textbf Spring 2014} Head-TA, CS10 (300 students)
    
    \cvitem {\textbf Fall 2013} TA, CS10 (300 students)
    
    \cvitem {\textbf Spring 2013} TA, CS194-23: The Art \& Science of Digital Photography (30 students); TA, CS10 (300 students)
    
    \cvitem {\textbf Fall 2012} TA, CS10 (240 students)
    
% \end{itemize}
