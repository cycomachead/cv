\section{Education}

\vspace{5pt}

\subsection{Academic Qualifications}

\vspace{5pt}

\cventry{2016}{MS Computer Science}{University of California, Berkeley}{Berkeley, CA}{\small{Advisor: Dr. Daniel Garcia}}{Thesis: Lambda: An Autograder for Snap!}

\vspace{5pt} % intentionally a bit larger..

\cventry{2015}{BA Computer Science}{University of California, Berkeley}{Berkeley, CA}{}{}
    
% \item{\cventry{Arg 1}{Arg 2}{Arg 3}{Arg 4}{Arg 5}{Arg 6}}
% arguments 3 to 6 can be left empty

\vspace{10pt}


% \subsection{Notable Projects}

% \vspace{6pt}

\textbf{Masters Thesis, 2016:} \textit{Lambda: Autograding For Snap!}

\vspace{3pt}

% TODO: Shrink and move up.
\small{Lambda is an autograding platform for Snap!, a blocks-based programming language. As an undergraduate, I contributed to designing the system, architecting code, and creating autograder tests for student projects. During my Master's Project, I created a server-side component for the autograder to support using LTI (Learning Tools Interoperability), an open educational standard, which allowed embedding the autograder into UC Berkeley's LMS. The autograder was used to trial a series of in-lab exercises, in comparison to the oral lab check-off questions students were answering.}
\linebreak
\small{\href{http://www2.eecs.berkeley.edu/Pubs/TechRpts/2018/EECS-2018-2.pdf}{http://www2.eecs.berkeley.edu/Pubs/TechRpts/2018/EECS-2018-2.pdf}}

\vspace{15pt}

% \vspace{10pt}