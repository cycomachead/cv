\section{Awards}
\vspace{6pt}

\begin{itemize}

  \setlength\itemsep{1em}

    \item UC Berkeley Lecturer Teaching Fellows, 2019-2020
    \vspace{2pt}
    \newline\small Year-long project on Parsons Puzzles in intro CS courses.
    \vspace{2pt}
    
    \item Best Lightning Talk Award, 49th ACM Technical Symposium on Computer Science Education, 2018
    \vspace{2pt}
    \newline\small Awarded for the lightning talk "IRT In 5 minutes".
    \vspace{2pt}

    \item Eugene L. Lawler Prize, UC Berkeley EECS, 2015
    \vspace{2pt}
    \newline\small{The Lawler Prize honors computer science undergraduates from disadvantaged groups (such as the disabled community and ethnic minorities) or a person who has surmounted unusual difficulties in pursuing a degree with demonstrated academic effort. This prize was established by the UC Berkeley Computer Science Division to honor the memory of colleague, Professor Eugene L. Lawler, who was an internationally recognized expert in mathematical theories of scheduling and resource allocation.}
    \vspace{2pt}
    \small \href{https://www2.eecs.berkeley.edu/Students/Awards/\#11}{https://www2.eecs.berkeley.edu/Students/Awards/\#11}
    
    \item EECS Distinguished Graduate Student Instructor Award, 2015
    \vspace{2pt}
    \newline
    \small{Faculty nominate the top nine percent of UC Berkeley EE or CS GSIs and Group Tutors from the semesters of the previous calendar year. From these, the EECS Student Awards committee selects one top EE and one CS GSI for the departmental award. Because EECS is a large department with about 180 GSIs per semester, it is a great honor to be selected as an Outstanding GSI.}
    \vspace{2pt}
    \small\href{https://www2.eecs.berkeley.edu/Students/Awards/\#13}{https://www2.eecs.berkeley.edu/Students/Awards/\#13}

\end{itemize}

\section{Grants and Gifts}
\vspace{6pt}

\begin{itemize}

  \setlength\itemsep{1em}

    \item{UC Berkeley Presidential Chair Fellows (2021-2022) Fox, Pamela, \me
    \newline
    }
    
    \item{Teach Access Grants (2021), \$5,000 \me
        \newline
        Adding accessibility content to CS169A
    }
    
    \item{NSF EAGER, Student Mission Control for the International Space Station (2021),
    \$298,944, Research Engineer (Dan Garcia co-PI)
    Funding to support development of an API, website, "Student Mission Control" interface, and curriculum modules centered on the data streaming out of the International Space Station. NSF Award \#2027260}

    \item{UC Berkeley College of Engineering, Course Adaptation and Remote Delivery, Learning, and Assessment: Developing Question Generators and MOOC-like videos \& quizzes for remote CS61C, CS10, and CS169A (2020), \$60,000, Dan Garcia co-PI
    Funding to build Question Generators and MOOC-like videos and quizzes for CS61C and CS10 for remote delivery, learning, and assessment.
    }
    
    \item{Hopper-Dean Foundation, Accelerating CS Diversity Programs Fund (2019), \$600,000, Dan Garcia PI
    The foundation granted \$3M to the department to fund diversity initiatives;  Summer salary was used to support our middle school curriculum, Spanish translation, software development and staff.
    }

    \item{Hopper-Dean Foundation (\$200,000). 2016. Researcher. (Daniel Garcia, PI.)}
    %You can find the full text of the award here: https://docs.google.com/document/d/1TuIuuHpOJu3LUEn6E9aJzKJVSJBH5b65OI6IEFleoYY/edit?usp=sharing
    \newline\small{For the support of diversity initiatives in CS to support high schools nationwide. The High School Initiative focuses on high school computer science teachers, led by Professor Dan Garcia. In response to NSF Program Manager Jan Cuny’s charge to change the face of computing by engaging and preparing 10,000 teachers to teach computer science courses, Professor Garcia and colleagues have developed our non-major computer science class CS10: The Beauty and Joy of Computing (BJC) into a course that is fully aligned with high school Academic Placement (AP) specification. The plan is to accomplish this objective via an edX Small Private Online Course (SPOC) experience, with the high school teacher in control. The course recently received national exposure when it had more women than men in an intro CS course for the first time since records were digitized.}

    \item{Google CS Engagement Award (\$5,000). 2015. Michael Ball.}
    \newline\small{This gift from Google was used to support CS10 and The Beauty and Joy of Computing. The funds have been used to develop Snap\textit{!} enhancements and help fund the Snap\textit{!} Cloud infrastructure.}
    
    \item{Google’s 3X in 3 Years (2015), \$900,000, Student  Researcher
    Authored By Dan Garcia. Funding to the department for 3-year project to grow undergraduate capacity and support diversity via our "Scaling Computer Science through Targeted Engagement" project. The three objectives are (1) Decrease the intro GPA gap between experienced and inexperienced students by 50\%, (2) Increase Software Engineering and UI Design enrollment by 500 total students/year, and (3) Increase the number of women and underrepresented minority CS majors by a factor of 3.}
    
    \item{NSF STEM-C BJC4NYC: Bringing the Beauty and Joy of Computing to the Largest School System in the US (\$7,874,876). 2014. Consultant.}
    \newline\small{For the development of curricular materials, based on the Beauty and Joy of Computing, for teaching CS Principles at the high school level using the Snap\textit{!} programming language. Development of the Snap\textit{!} language, including the cloud back-end, and BJC curriculum software. l During the project, 100 high school teachers in New York City were trained to teach BJC, and early participants become teacher-trainers who worked with later participants. The teachers involved become part of a Community of Practice that continues to provide support for the teacher cohorts. I worked with EDC curriculum developers to review and suggest changes, developed BJCx (with auto-grading features) for teachers as a small private online course (SPOC), and (3) supported BJC teachers online.}
    
    \item{edX (\$50,000). 2014. Staff.}
    %You can find the actual proposal here: https://drive.google.com/file/d/0B4KuCtIkhB7QWmJvVXdESy1Gb0U/view?usp=sharing
    \newline\small{For the development of BJCx, a Computer Science Principles edX MOOC to be offered (1) as a synchronous Small Private Online Course (SPOC) for high schools, (2) a synchronous open-to-all MOOC for others, and (3) an asynchronous self-study course. The development concluded in creating four sub-MOOCs, which together was the sum of the BJC offering. I supported Snap\textit{!} cloud integration, developed autograding (remote grading of Python and Snap\textit{!} assignments and lab activities in a remove server) and an automated lab content builder, and supported BJC students and teachers in an online forum.}

\end{itemize}
